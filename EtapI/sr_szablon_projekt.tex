% !TeX encoding = UTF-8
% !TeX spellcheck = pl_PL


% $Id:$

%Author: Wojciech Domski
%Szablon do ząłożeń projektowych, raportu i dokumentacji z steorwników robotów
%Wersja v.1.0.0
%

%% Konfiguracja:
\newcommand{\kurs}{Sterowniki robot\'{o}w}
\newcommand{\formakursu}{Projekt}

%odkomentuj właściwy typ projektu
\newcommand{\doctype}{Za\l{}o\.{z}enia projektowe}
%\newcommand{\doctype}{Raport}
%\newcommand{\doctype}{Dokumentacja}

%wpisz nazwę projektu
\newcommand{\projectname}{Sterowany Pochyleniem R\k{e}ki Pojazd Prawie Autonomiczny}

%wpisz akronim projektu
\newcommand{\acronim}{S.P.R.P.P.A}

%zmaiast X wpisz numer grupy projektowej
\newcommand{\nrgrupy}{6}
%wpisz Imię i nazwisko oraz numer albumu
\newcommand{\osobaA}{Patrycjusz \textsc{Augu\.{s}cik}, 226523}
%w przypadku projektu jednoosobowego usuń zawartość nowej komendy
\newcommand{\osobaB}{Maciej \textsc{Kajdak}, 226256}

%wpisz termin w formie, jak poniżej dzień, parzystość, godzina
\newcommand{\termin}{wtTP11}

%wpisz imię i nazwisko prowadzącego
\newcommand{\prowadzacy}{mgr in\.{z}. Wojciech \textsc{Domski}}

\documentclass[10pt, a4paper]{article}
% W nawiasie klamrowym podana jest klasa dokumentu. Standardowe klasy artykułu
% to: article, amsart, scrartcl, artikel1, artikel2, artikel3.
% W nawiasie prostokątnym deklarowane są opcje dokumentu. Zamiast 10pt
% można podać 11pt lub 12pt. Dokument w dwóch kolumnach uzyskuje się po
% wpisaniu opcji twocolumn, 

\usepackage[MeX]{polski}
\usepackage[utf8]{inputenc}
\usepackage{caption}
\usepackage{float}
\usepackage{lscape}

\include{preambula}
	
\begin{document}

\def\tablename{Tabela}	%zmienienie nazwy tabel z Tablica na Tabela

\begin{titlepage}
	\begin{center}
		\textsc{\LARGE \formakursu}\\[1cm]		
		\textsc{\Large \kurs}\\[0.5cm]		
		\rule{\textwidth}{0.08cm}\\[0.4cm]
		{\huge \bfseries \doctype}\\[1cm]
		{\huge \bfseries \projectname}\\[0.5cm]
		{\huge \bfseries \acronim}\\[0.4cm]
		\rule{\textwidth}{0.08cm}\\[1cm]
		
		\begin{flushright} \large
		\emph{Skład grupy (\nrgrupy):}\\
		\osobaA\\
		\osobaB\\[0.4cm]
		
		\emph{Termin: }\termin\\[0.4cm]

		\emph{Prowadzący:} \\
		\prowadzacy \\
		
		\end{flushright}
		
		\vfill
		
		{\large \today}
	\end{center}	
\end{titlepage}

\newpage
\tableofcontents
\newpage

	\section{Opis projektu i założenia projektowe}
\subsection{Projekt nadajnika wykorzystującego akcelerometr do sterowania pojazdem kołowym}
Projekt zakłada wykorzystanie akcelerometra dostępnego na płytce rozwojowej STM32L476 Discovery do sterowania pojazdem kołowym. Jest to moduł MEMS LSM303CTR z wbudowanym akcelerometrem i magnetometrem. Mikrokontroler będzie łączył się z modułem za pomocą szeregowego  interfejsu urządzeń peryferyjnych -- SPI w trybie Master Receives Only. Komunikacja między samochodzikiem a płytką odbywać się będzie za pomocą układu WiFi + Bluetooth BLE ESP-WROOM-32 - SMD. Z modułem mikrokontroler będzie się łączył dzięki komunikacji UART. W naszym projekcie zostanie wykorzystany tylko moduł bluetooth. Moduł ten w tej części projektu będzie pełnił rolę nadajnika. Pojazd będzie się poruszał w kierunku wskazanym przez dłoń sterującego. Aby połączyć się z samochodzikiem, należy trzymać w dłoni płytkę uruchomieniową, która będzie się łączyć z samochodzikiem automatycznie. 
Możliwości ruchu pojazdu:
\\* -- do przodu
\\* -- do tyłu
\\* -- w lewo
\\* -- w prawo
\\* Prędkość samochodzika będzie uzależniona od szybkości ruchów ręki.


\subsection{Projekt odbiornika i pojazdu kołowego}
Projekt zakłada wykorzystanie układu WiFi + Bluetooth BLE ESP-WROOM-32 - SMD. Z tego modułu zostanie wykorzystany tylko moduł bluetooth jako odbiornik informacji z nadajnika. Do zbudowania pojazdu zostanie wykorzystany stary samochodzik - zabawka. W celu ulepszenia samochodu - zamontujemy nowe silniczki komutatorowe prądu stałego. Pojazd ten będzie mógł osiągnąć dużą prędkość dzięki przekładni 2:1. Wmontujemy również czujniki odległości, a zadaniem pojazdu będzie natychmiastowe zatrzymanie się w przypadku napotkania przeszkody lub w momencie utraty połączenia bluetooth z nadajnikiem.
W opisywanym samochodziku wykorzystamy napęd na przednią oś, silniki zostaną połączone z mostkami H, a całością będzie sterować mikrokontroler.

\subsection{Funkcjonalności dodatkowe}
Nieobowiązkowo projekt zakłada dodanie funkcjonalności zmiany źródła sterowania. Przełączenie sterowania ma się opierać o dodatkową płytkę Raspberry Pi, dzięki której możliwe będzie przetwarzanie obrazu z kamery zamontowanej na samochodziku. Tym sposobem samochodzik miałby się poruszać za określonym przedmiotem (np. małą piłką w mocno jaskrawym kolorze). Przełączenie sterowania miałoby nastąpić po jawnym wybraniu odpowiedniej opcji na płytce Discovery, a do tego celu pomocne będzie użycie wyświetlacza LCD. 




%
%\begin{lstlisting}[tabsize=2]
%int foo(void){
%	return 2;
%}
%\end{lstlisting}
%
%\begin{table}[H]
%	\centering
%	\begin{tabular}{|l|l|}
%		\hline
%		Kolumna A	&	Kolumna B\\
%		\hline
%		1 A			&	1B\\
%		2 A			&	2B\\
%		3 A			&	3B\\
%		4 A			&	4B\\
%		\hline
%	\end{tabular}
%	\caption{Rzędy i kolumny}
%	\label{tab:Tabelka}
%\end{table}
%
%\begin{figure}[H]
%	\centering
%	%poniższą linię odkomentuj oraz podaj ściezkę do obrazka
%	%\includegraphics[width=0.8\textwidth]{grafika/rysunek.png}
%	\caption{Rysunek2}
%	\label{fig:Rysunek2}
%\end{figure}
%
%Przykładowy wzór (\ref{eq:Wzor}):
%\begin{equation}
%\label{eq:Wzor}
%\Theta = \int_t^{t+dt} \omega \, dt.	
%\end{equation}
%
%Cytowanie pracy \cite{Wd14}.

\newpage
\section{Organizacja pracy}
\subsection{Harmonogram zadań}
Harmonogram pracy zespołu wraz z datami terminu został przedstawiony na rysunku \ref{fig:Harmonogram}. Na podstawie danych z harmonogramu rozpoczęto pracę nad rozdziałem pomniejszych zadań. Harmonogram został wygenerowany przy pomocy programu Ganttproject.

\begin{figure}[H]
	\centering
	\includegraphics[width=0.8\textwidth]{Harm.png}
	\caption{Harmonogram pracy}
	\label{fig:Harmonogram}
\end{figure}


\newpage
\subsection{Podział pracy}
Podział pracy pomiędzy członków grupy został przedstawiony w tabeli \ref{tab:Tabela rozkladu pracy}.

\begin{table}[H]
	\centering
	\begin{tabular}{|p{7.8cm}|p{7.8cm}|} \hline
%	\begin{tabular}{|l|l|}

		\textbf{Patrycjusz } & \textbf{Maciej} \\ \hline \hline
		dobór elementów potrzebnych do realizacji projektu & konfiguracja Cube \\ \hline
		projekt 3D pojazdu & algorytm sterowania pojazdem za pomocą akcelerometru na płytce Discovery \\ \hline
		wydrukowanie modelu na drukarce 3D & testy poprawności działania algorytmu podłączając pojazd do płytki za pomocą kabla \\ \hline
		projekt elektroniki i płytki PCB &  algorytm sterowania pojazdem w momencie gdy zostanie wykryta przeszkoda  \\ \hline
		polutowanie układu oraz testy poprawności działania &  testy poprawności działania algorytmu podłączając pojazd do płytki za pomocą kabla \\ \hline
		podłączenie elektroniki i montaż elementów mechanicznych &--------------- \\ \hline
		\multicolumn{2}{|c|}{rozwój modułu komunikacji} \\ \hline
		\multicolumn{2}{|c|}{testy poprawności działania robota} \\ \hline
	\end{tabular}
	\caption{Tabela rozkładu zadań}	
	\label{tab:Tabela rozkladu pracy}
\end{table}

\noindent Biorąc pod uwagę przedstawiony rozkład pracy oraz jej harmonogram uwzględniający terminy oddania raportów z poszczególnych etapów projektu przystąpiono do stworzenia diagramu Gantta niniejszego projektu.

\newpage
\begin{landscape}
\subsection{Diagram Gantta}
Na podstawie stworzonego harmonogramu oraz rozkładu pracy wygenerowano diagram Gantta korzystając z programu Ganttproject.
Diagram przedstawiono na rysunku \ref{fig:Gantt}.


\begin{figure}[H]
	\centering
	\includegraphics[scale=0.6, keepaspectratio]{GanttSR.png}
	\caption{Diagram Gantta}
	\label{fig:Gantt}
\end{figure}
\end{landscape}
%\newpage
%\section{Podsumowanie}


%\newpage
%	\section{Schematy elektroniczne}
%	\begin{figure}[H]
%		\centering	
%		\includegraphics[scale=0.5, width=\textwidth ,height=\textheight ,keepaspectratio]{STM32.png}
%		\caption{Schemat polaczen STM32F103RBT6}
%	\end{figure}
%
%\begin{figure}[H]
%	\centering
%	\includegraphics[scale=0.5, width=\textwidth ,height=\textheight ,keepaspectratio]{czujniki.png}
%	\caption{Schemat połączeń programatora, czujników odległości oraz mostka H}
%\end{figure}


%\addcontentsline{toc}{section}{Bibilografia}
%\bibliography{bibliografia}
%\bibliographystyle{plain}


\newpage
\listoffigures
\newpage
\listoftables


\end{document}







































